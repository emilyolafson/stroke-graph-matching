
@ARTICLE{Shen2013-zn,
	title    = "Groupwise whole-brain parcellation from resting-state {fMRI} data
	for network node identification",
	author   = "Shen, X and Tokoglu, F and Papademetris, X and Constable, R T",
	abstract = "In this paper, we present a groupwise graph-theory-based
	parcellation approach to define nodes for network analysis. The
	application of network-theory-based analysis to extend the
	utility of functional MRI has recently received increased
	attention. Such analyses require first and foremost a reasonable
	definition of a set of nodes as input to the network analysis. To
	date many applications have used existing atlases based on
	cytoarchitecture, task-based fMRI activations, or anatomic
	delineations. A potential pitfall in using such atlases is that
	the mean timecourse of a node may not represent any of the
	constituent timecourses if different functional areas are
	included within a single node. The proposed approach involves a
	groupwise optimization that ensures functional homogeneity within
	each subunit and that these definitions are consistent at the
	group level. Parcellation reproducibility of each subunit is
	computed across multiple groups of healthy volunteers and is
	demonstrated to be high. Issues related to the selection of
	appropriate number of nodes in the brain are considered. Within
	typical parameters of fMRI resolution, parcellation results are
	shown for a total of 100, 200, and 300 subunits. Such
	parcellations may ultimately serve as a functional atlas for fMRI
	and as such three atlases at the 100-, 200- and 300-parcellation
	levels derived from 79 healthy normal volunteers are made freely
	available online along with tools to interface this atlas with
	SPM, BioImage Suite and other analysis packages.",
	journal  = "Neuroimage",
	volume   =  82,
	pages    = "403--415",
	month    =  nov,
	year     =  2013,
	keywords = "Functional MRI; Graph-theory-based parcellation; Network
	analysis; Resting-state connectivity; Whole-brain atlas",
	language = "en"
}

@UNPUBLISHED{Mueller2020-xg,
	title    = "Dynamic community detection reveals transient reorganization of
	functional brain networks across a female menstrual cycle",
	author   = "Mueller, Joshua M and Pritschet, Laura and Santander, Tyler and
	Taylor, Caitlin M and Grafton, Scott T and Jacobs, Emily Goard
	and Carlson, Jean M",
	abstract = "Sex steroid hormones have been shown to alter regional brain
	activity, but the extent to which they modulate connectivity
	within and between large-scale functional brain networks over
	time has yet to be characterized. Here, we applied dynamic
	community detection techniques to data from a highly sampled
	female with 30 consecutive days of brain imaging and venipuncture
	measurements to characterize changes in resting-state community
	structure across the menstrual cycle. Four stable functional
	communities were identified consisting of nodes from visual,
	default mode, frontal control, and somatomotor networks. Limbic,
	subcortical, and attention networks exhibited higher than
	expected levels of nodal flexibility, a hallmark of
	between-network integration and transient functional
	reorganization. The most striking reorganization occurred in a
	default mode subnetwork localized to regions of the prefrontal
	cortex, coincident with peaks in serum levels of estradiol,
	luteinizing hormone, and follicle stimulating hormone. Nodes from
	these regions exhibited strong intra-network increases in
	functional connectivity, leading to a split in the stable default
	mode core community and the transient formation of a new
	functional community. Probing the spatiotemporal basis of human
	brain--hormone interactions with dynamic community detection
	suggests that ovulation results in a temporary, localized
	patterns of brain network reorganization. Author Summary Sex
	steroid hormones influence the central nervous system across
	multiple spatiotemporal scales. Estrogen and progesterone
	concentrations rise and fall throughout the menstrual cycle, but
	it remains poorly understood how day-to-day fluctuations in
	hormones shape human brain dynamics. Here, we assessed the
	structure and stability of resting-state brain network activity
	in concordance with serum hormone levels from a female who
	underwent fMRI and venipuncture for 30 consecutive days. Our
	results reveal that while network structure is largely stable
	over the menstrual cycle, there is temporary reorganization of
	several largescale functional brain networks during the ovulatory
	window. In particular, a default mode subnetwork exhibits
	increased connectivity with itself and with regions from
	temporoparietal and limbic networks, providing novel perspective
	into brain-hormone interactions. \#\#\# Competing Interest
	Statement The authors have declared no competing interest.",
	journal  = "Cold Spring Harbor Laboratory",
	pages    = "2020.06.29.178152",
	month    =  jun,
	year     =  2020,
	language = "en"
}

@ARTICLE{Conte2004-ro,
	title     = "Thirty Years Of Graph Matching In Pattern Recognition",
	author    = "Conte, Donatello and Foggia, Pasquale and Sansone, Carlo and
	Vento, Mario",
	abstract  = "PDF | A recent paper posed the question: ``Graph Matching: What
	are we really talking about?''. Far from providing a definite
	answer to that question, in... | Find, read and cite all the
	research you need on ResearchGate",
	journal   = "Int. J. Pattern Recognit Artif Intell.",
	publisher = "World Scientific Publishing",
	volume    =  18,
	number    =  3,
	pages     = "265--298",
	month     =  may,
	year      =  2004
}

@ARTICLE{Pool2018-px,
	title    = "Network dynamics engaged in the modulation of motor behavior in
	stroke patients",
	author   = "Pool, Eva-Maria and Leimbach, Martha and Binder, Ellen and
	Nettekoven, Charlotte and Eickhoff, Simon B and Fink, Gereon R
	and Grefkes, Christian",
	abstract = "Stroke patients with motor deficits typically feature enhanced
	neural activity in several cortical areas when moving their
	affected hand. However, also healthy subjects may show higher
	levels of neural activity in tasks with higher motor demands.
	Therefore, the question arises to what extent stroke-related
	overactivity reflects performance-level-associated recruitment of
	neural resources rather than stroke-induced neural
	reorganization. We here investigated which areas in the lesioned
	brain enable the flexible adaption to varying motor demands
	compared to healthy subjects. Accordingly, eleven well-recovered
	left-hemispheric chronic stroke patients were scanned using
	functional magnetic resonance imaging. Motor system activity was
	assessed for fist closures at increasing movement frequencies
	performed with the affected/right or unaffected/left hand. In
	patients, an increasing movement rate of the affected hand was
	associated with stronger neural activity in ipsilesional/left
	primary motor cortex (M1) but unlike in healthy controls also in
	contralesional/right dorsolateral premotor cortex (PMd) and
	contralesional/right superior parietal lobule (SPL). Connectivity
	analyses using dynamic causal modeling revealed stronger coupling
	of right SPL onto affected/left M1 in patients but not in
	controls when moving the affected/right hand independent of the
	movement speed. Furthermore, coupling of right SPL was positively
	coupled with the ``active'' ipsilesional/left M1 when stroke
	patients moved their affected/right hand with increasing movement
	frequency. In summary, these findings are compatible with a
	supportive role of right SPL with respect to motor function of
	the paretic hand in the reorganized brain.",
	journal  = "Hum. Brain Mapp.",
	volume   =  39,
	number   =  3,
	pages    = "1078--1092",
	month    =  mar,
	year     =  2018,
	keywords = "dynamic causal modeling; effective connectivity; movement
	frequency; reorganization; superior parietal lobe",
	language = "en"
}

@ARTICLE{Rehme2013-ap,
	title    = "Cerebral network disorders after stroke: evidence from
	imaging-based connectivity analyses of active and resting brain
	states in humans",
	author   = "Rehme, Anne K and Grefkes, Christian",
	abstract = "Stroke causes a sudden disruption of physiological brain function
	which leads to impairments of functional brain networks involved
	in voluntary movements. In some cases, the brain has the
	intrinsic capacity to reorganize itself, thereby compensating for
	the disruption of motor networks. In humans, such reorganization
	can be investigated in vivo using neuroimaging. Recent
	developments in connectivity analyses based on functional
	neuroimaging data have provided new insights into the network
	pathophysiology underlying neurological symptoms. Here we review
	recent neuroimaging studies using functional resting-state
	correlations, effective connectivity models or graph theoretical
	analyses to investigate changes in neural motor networks and
	recovery after stroke. The data demonstrate that network
	disturbances after stroke occur not only in the vicinity of the
	lesion but also between remote cortical areas in the affected and
	unaffected hemisphere. The reorganization of motor networks
	encompasses a restoration of interhemispheric functional
	coherence in the resting state, particularly between the primary
	motor cortices. Furthermore, reorganized neural networks feature
	strong excitatory interactions between fronto-parietal areas and
	primary motor cortex in the affected hemisphere, suggesting that
	greater top-down control over primary motor areas facilitates
	motor execution in the lesioned brain. In addition, there is
	evidence that motor recovery is accompanied by a more random
	network topology with reduced local information processing. In
	conclusion, Stroke induces changes in functional and effective
	connectivity within and across hemispheres which relate to motor
	impairments and recovery thereof. Connectivity analyses may hence
	provide new insights into the pathophysiology underlying
	neurological deficits and may be further used to develop novel,
	neurobiologically informed treatment strategies.",
	journal  = "J. Physiol.",
	volume   =  591,
	number   =  1,
	pages    = "17--31",
	month    =  jan,
	year     =  2013,
	language = "en"
}

@ARTICLE{Urbin2014-iq,
	title    = "Resting-state functional connectivity and its association with
	multiple domains of upper-extremity function in chronic stroke",
	author   = "Urbin, M A and Hong, Xin and Lang, Catherine E and Carter, Alex R",
	abstract = "BACKGROUND: Recent work has shown that resting-state functional
	connectivity (rsFC) between homotopic, motor-related brain
	regions is associated with upper-extremity control early after
	stroke. OBJECTIVES: This study examined various patterns of rsFC
	in chronic stroke, a time at which extensive neural
	reorganization has occurred. Associations between homotopic
	somatomotor connectivity and clinical measures, representing
	separate domains of upper-extremity function, were determined.
	METHODS: A total of 19 persons $\geq$6 months poststroke
	participated. Four connectivity patterns within a somatomotor
	network were quantified using functional magnetic resonance
	imaging. Upper-extremity gross muscle activation, control, and
	real-world use were evaluated with the Motricity Index, Action
	Research Arm Test, and accelerometry, respectively. RESULTS:
	Connectivity between homotopic regions was stronger than that in
	the contralesional and ipsilesional hemispheres. No differences
	in connectivity strength were noted between homotopic pairs,
	indicating that a specific brain structure was not driving
	somatomotor network connectivity. Homotopic connectivity was
	significantly associated with both upper-extremity control (r =
	0.53; P= .02) and real-world use (r = 0.54; P= .02); however,
	there was no association with gross muscle activation (r = 0.23;
	P=.34). The combination of clinical measures accounted for 40\%
	of the variance in rsFC (= .05). CONCLUSIONS: The results
	reported here expand on previous findings, indicating that
	homotopic rsFC persists in chronic stroke and discriminates
	between varying levels of upper-extremity control and real-world
	use. Further work is needed to evaluate its adequacy as a
	biomarker of motor recovery following stroke.",
	journal  = "Neurorehabil. Neural Repair",
	volume   =  28,
	number   =  8,
	pages    = "761--769",
	month    =  oct,
	year     =  2014,
	keywords = "hemiparesis; motor control; resting-state functional
	connectivity; stroke",
	language = "en"
}

@ARTICLE{Carter2010-er,
	title    = "Resting interhemispheric functional magnetic resonance imaging
	connectivity predicts performance after stroke",
	author   = "Carter, Alex R and Astafiev, Serguei V and Lang, Catherine E and
	Connor, Lisa T and Rengachary, Jennifer and Strube, Michael J and
	Pope, Daniel L W and Shulman, Gordon L and Corbetta, Maurizio",
	abstract = "OBJECTIVE: Focal brain lesions can have important remote effects
	on the function of distant brain regions. The resulting network
	dysfunction may contribute significantly to behavioral deficits
	observed after stroke. This study investigates the behavioral
	significance of changes in the coherence of spontaneous activity
	in distributed networks after stroke by measuring resting state
	functional connectivity (FC) using functional magnetic resonance
	imaging. METHODS: In acute stroke patients, we measured FC in a
	dorsal attention network and an arm somatomotor network, and
	determined the correlation of FC with performance obtained in a
	separate session on tests of attention and motor function. In
	particular, we compared the behavioral correlation with
	intrahemispheric FC to the behavioral correlation with
	interhemispheric FC. RESULTS: In the attention network,
	disruption of interhemispheric FC was significantly correlated
	with abnormal detection of visual stimuli (Pearson r with field
	effect = -0.624, p = 0.002). In the somatomotor network,
	disruption of interhemispheric FC was significantly correlated
	with upper extremity impairment (Pearson r with contralesional
	Action Research Arm Test = 0.527, p = 0.036). In contrast,
	intrahemispheric FC within the normal or damaged hemispheres was
	not correlated with performance in either network. Quantitative
	lesion analysis demonstrated that our results could not be
	explained by structural damage alone. INTERPRETATION: These
	results suggest that lesions cause state changes in the
	spontaneous functional architecture of the brain, and constrain
	behavioral output. Clinically, these results validate using FC
	for assessing the health of brain networks, with implications for
	prognosis and recovery from stroke, and underscore the importance
	of interhemispheric interactions.",
	journal  = "Ann. Neurol.",
	volume   =  67,
	number   =  3,
	pages    = "365--375",
	month    =  mar,
	year     =  2010,
	language = "en"
}

@ARTICLE{Carrera2014-ah,
	title    = "Diaschisis: past, present, future",
	author   = "Carrera, Emmanuel and Tononi, Giulio",
	abstract = "After a century of false hopes, recent studies have placed the
	concept of diaschisis at the centre of the understanding of brain
	function. Originally, the term 'diaschisis' was coined by von
	Monakow in 1914 to describe the neurophysiological changes that
	occur distant to a focal brain lesion. In the following decades,
	this concept triggered widespread clinical interest in an attempt
	to describe symptoms and signs that the lesion could not fully
	explain. However, the first imaging studies, in the late 1970s,
	only partially confirmed the clinical significance of diaschisis.
	Focal cortical areas of diaschisis (i.e. focal diaschisis)
	contributed to the clinical deficits after subcortical but only
	rarely after cortical lesions. For this reason, the concept of
	diaschisis progressively disappeared from the mainstream of
	research in clinical neurosciences. Recent evidence has
	unexpectedly revitalized the notion. The development of new
	imaging techniques allows a better understanding of the
	complexity of brain organization. It is now possible to reliably
	investigate a new type of diaschisis defined as the changes of
	structural and functional connectivity between brain areas
	distant to the lesion (i.e. connectional diaschisis). As opposed
	to focal diaschisis, connectional diaschisis, focusing on
	determined networks, seems to relate more consistently to the
	clinical findings. This is particularly true after stroke in the
	motor and attentional networks. Furthermore, normalization of
	remote connectivity changes in these networks relates to a better
	recovery. In the future, to investigate the clinical role of
	diaschisis, a systematic approach has to be considered. First,
	emerging imaging and electrophysiological techniques should be
	used to precisely map and selectively model brain lesions in
	human and animals studies. Second, the concept of diaschisis must
	be applied to determine the impact of a focal lesion on new
	representations of the complexity of brain organization. As an
	example, the evaluation of remote changes in the structure of the
	connectome has so far mainly been tested by modelization of focal
	brain lesions. These changes could now be assessed in patients
	suffering from focal brain lesions (i.e. connectomal diaschisis).
	Finally, and of major significance, focal and non-focal
	neurophysiological changes distant to the lesion should be the
	target of therapeutic strategies. Neuromodulation using
	transcranial magnetic stimulation is one of the most promising
	techniques. It is when this last step will be successful that the
	concept of diaschisis will gain all the clinical respectability
	that could not be obtained in decades of research.",
	journal  = "Brain",
	volume   =  137,
	number   = "Pt 9",
	pages    = "2408--2422",
	month    =  sep,
	year     =  2014,
	keywords = "brain function; brain organization; diaschisis; stroke",
	language = "en"
}

@ARTICLE{Saia2009-ik,
	title    = "Progressive stroke in pontine infarction",
	author   = "Saia, V and Pantoni, L",
	abstract = "OBJECTIVE: The pathogenesis of isolated pontine infarctions is
	still unclear, being attributed both to small or large vessel
	disease. The extension of infarcted tissue to the pons surface
	has been indicated as a possible marker of basilar branch
	atheromatous disease and some neuroimaging evidence confirms this
	finding. METHODS: On the basis of Kim's et al., study, we
	performed a revision of the literature addressing this topic.
	RESULTS: Several authors confirm an association between basilar
	artery branch disease and isolated pontine infarction; moreover,
	the enlargement of pontine lesion seems to be associated with
	neurological worsening. We therefore performed a brief analysis
	of possible mechanisms of progression. CONCLUSIONS: Prospective
	studies could be useful to evaluate predictors of neurological
	worsening in pontine stroke. Improvement of neuroimaging
	techniques is needed for a deeper comprehension of the
	etiopathogenesis of isolated pontine infarction.",
	journal  = "Acta Neurol. Scand.",
	volume   =  120,
	number   =  4,
	pages    = "213--215",
	month    =  oct,
	year     =  2009,
	language = "en"
}

@ARTICLE{Wang2019-jr,
	title    = "Cerebral blood flow features in chronic subcortical stroke:
	Lesion location-dependent study",
	author   = "Wang, Caihong and Miao, Peifang and Liu, Jingchun and Wei, Sen
	and Guo, Yafei and Li, Zhen and Zheng, Dandan and Cheng,
	Jingliang",
	abstract = "We investigated the influence of lesion location on cerebral
	blood flow (CBF) in chronic subcortical stroke patients.
	Three-dimensional pseudocontinuous arterial spin labeling was
	employed to obtain CBF images in normal controls (NC) and
	patients with left hemisphere subcortical infarctions involving
	motor pathways. Stroke patients were divided into two subgroups
	based on the infarction location (basal ganglia (BS) or pontine
	(PS). We mapped CBF alterations in a voxel-wise manner and
	compared them to detect differences among groups with
	height-level false discovery rate correction. Regions with
	significant group differences were extracted to perform post hoc
	analyses among the BS, PS and NC groups using a general linear
	model with age, gender, years of education, and interval after
	stroke as covariates. The BS group displayed significantly
	increased CBF in the contralesional putamen relative to NC and
	significantly decreased CBF in the ipsilesional sensorimotor
	cortex, ipsilesional thalamus and contralesional cerebellum. The
	PS group displayed significantly increased CBF in the
	contralesional inferior frontal gyrus relative to both the NC and
	BS groups. Nevertheless, the PS group showed significantly
	decreased CBF mainly in the cerebellum. Our results suggest
	different alteration patterns of CBF in chronic stroke patients
	with different infarct locations within subcortical motor
	pathways, potentially providing important information for the
	initiation of individualized rehabilitation strategies for
	subcortical stroke patients involving different infarct types.",
	journal  = "Brain Res.",
	volume   =  1706,
	pages    = "177--183",
	month    =  mar,
	year     =  2019,
	keywords = "Arterial spin labeling; Cerebral blood flow; Motor pathways;
	Subcortical stroke",
	language = "en"
}

@ARTICLE{Liu2019-mj,
	title    = "Network change in the ipsilesional cerebellum is correlated with
	motor recovery following unilateral pontine infarction",
	author   = "Liu, G and Tan, S and Peng, K and Dang, C and Xing, S and Xie, C
	and Zeng, J",
	abstract = "BACKGROUND AND PURPOSE: Patients with acute pontine infarcts
	generally have good short-term motor outcomes; however, the
	mechanisms underlying this recovery of function remain unclear.
	METHODS: Twenty well-recovered patients with acute pontine
	infarcts and 20 well-recovered patients with acute
	striato-capsular infarcts were recruited. Fugl-Meyer assessment
	and resting-state functional magnetic resonance imaging were
	performed 1, 4 and 12 weeks after onset. Patients were further
	assigned to better and worse recovery subgroups according to the
	degree of motor recovery at the twelfth week after stroke.
	Voxel-wise degree centrality (DC)-behavior correlation analysis
	was used to identify brain regions related to changes in motor
	function within 12 weeks after stroke. RESULTS: A significant
	correlation was found between DC and Fugl-Meyer scores within 12
	weeks in the ipsilesional cerebellar crus I and crus II in
	patients with pontine infarction and in the ipsilesional middle
	temporal gyrus in patients with striato-capsular infarction (all
	P < 0.001, AlphaSim corrected). The mean DC in these areas was
	higher both in the better and worse recovery subgroups at the
	fourth than at the first week (all P < 0.05). In addition, the
	mean DC values in these areas were higher in patients with better
	recovery at the twelfth than at the first week (P < 0.05), but
	such change was not found in patients with worse recovery.
	CONCLUSIONS: These results indicate that network changes in the
	ipsilesional cerebellum are correlated with motor recovery
	following pontine infarction. Motor recovery mechanisms may vary
	between pontine and striato-capsular infarcts, according to
	lesion location.",
	journal  = "Eur. J. Neurol.",
	volume   =  26,
	number   =  10,
	pages    = "1266--1273",
	month    =  oct,
	year     =  2019,
	keywords = "Fugl-Meyer assessment; degree centrality; magnetic resonance
	imaging; motor recovery; pontine infarction",
	language = "en"
}

@ARTICLE{Hordacre2021-ct,
	title    = "Fronto-parietal involvement in chronic stroke motor performance
	when corticospinal tract integrity is compromised",
	author   = "Hordacre, Brenton and Lotze, Mart{\'\i}n and Jenkinson, Mark and
	Lazari, Alberto and Barras, Christen D and Boyd, Lara and
	Hillier, Susan",
	abstract = "BACKGROUND: Preserved integrity of the corticospinal tract (CST)
	is a marker of good upper-limb behavior and recovery following
	stroke. However, there is less understanding of neural mechanisms
	that might help facilitate upper-limb motor recovery in stroke
	survivors with extensive CST damage. OBJECTIVE: The purpose of
	this study was to investigate resting state functional
	connectivity in chronic stroke survivors with different levels of
	CST damage and to explore neural correlates of greater upper-limb
	motor performance in stroke survivors with compromised
	ipsilesional CST integrity. METHODS: Thirty chronic stroke
	survivors (24 males, aged 64.7 $\pm$ 10.8 years) participated in
	this study. Three experimental sessions were conducted to: 1)
	obtain anatomical (T1, T2) structural (diffusion) and functional
	(resting state) MRI sequences, 2) determine CST integrity with
	transcranial magnetic stimulation (TMS) and conduct assessments
	of upper-limb behavior, and 3) reconfirm CST integrity status.
	Participants were divided into groups according to the extent of
	CST damage. Those in the extensive CST damage group did not show
	TMS evoked responses and had significantly lower ipsilesional
	fractional anisotropy. RESULTS: Of the 30 chronic stroke
	survivors, 12 were categorized as having extensive CST damage.
	Stroke survivors with extensive CST damage had weaker functional
	connectivity in the ipsilesional sensorimotor network and greater
	functional connectivity in the ipsilesional fronto-parietal
	network compared to those with preserved CST integrity. For
	participants with extensive CST damage, improved motor
	performance was associated with greater functional connectivity
	of the ipsilesional fronto-parietal network and higher fractional
	anisotropy of the ipsilesional rostral superior longitudinal
	fasciculus. CONCLUSIONS: Stroke survivors with extensive CST
	damage have greater resting state functional connectivity of an
	ipsilesional fronto-parietal network that appears to be a
	behaviorally relevant neural mechanism that improves upper-limb
	motor performance.",
	journal  = "Neuroimage Clin",
	volume   =  29,
	pages    = "102558",
	month    =  jan,
	year     =  2021,
	keywords = "Functional connectivity; Magnetic resonance imaging; Motor evoked
	potential; Motor skills; Neurological rehabilitation; Stroke",
	language = "en"
}

@ARTICLE{Duncan2000-uj,
	title    = "Defining post-stroke recovery: implications for design and
	interpretation of drug trials",
	author   = "Duncan, P W and Lai, S M and Keighley, J",
	abstract = "Measurement of stroke recovery is complex because definition of
	successful recovery is highly variable across measures and
	cut-off points for defining successful outcomes vary. The purpose
	of this paper is to describe patterns of recovery in stroke
	patients of varying severity when different measures are used and
	when different cut-off points are selected. 459 individuals
	enrolled in a prospective cohort study were assessed within 14
	days post stroke and re-evaluated at 1, 3, and 6 months. Recovery
	was assessed using the NIH Stroke Scale, the Fugl-Meyer
	Assessment of Motor Recovery, the Barthel Index of Activities of
	Daily Living, the Physical Function Index of the SF-36, and the
	Modified Rankin Outcome Scale. Subjects also defined their
	preference (utility) for their current health state with a
	time-trade off question. We compared patterns of recovery using
	the different measures and varying the cut-off points for
	defining successful recovery. The percentage of patients who are
	believed to have recovered depends on how recovery is defined. If
	recovery is defined at the disability level (Barthel > 90), the
	majority 57.3\% of stroke survivors experience a full recovery.
	Fewer individuals are considered to be fully recovered if
	impairments are measured (NIH 90, 36.8\%. Less than 25\% of
	stroke survivors are considered recovered if recovery is defined
	relative to reported prior function in higher levels of physical
	activity. Shifting the definition of recovery on the modified
	Rankin scale from </= 1 to </= 2 shifts the percentage of those
	deemed recovered from </= 25\% to 53.8\%. In designing drug
	trials the methods for defining stroke recovery should be
	carefully considered. If recovery is defined in terms of
	disability, a higher proportion of the placebo group will achieve
	the outcome than if impairments are used to define recovery. The
	benchmarks for recovery in minor strokes must include measures of
	higher functioning (e.g. the SF-36 physical functioning index or
	a Rankin 0 (no symptoms).",
	journal  = "Neuropharmacology",
	volume   =  39,
	number   =  5,
	pages    = "835--841",
	month    =  mar,
	year     =  2000,
	language = "en"
}

@ARTICLE{Wei2020-gj,
	title    = "Disrupted Regional Cerebral Blood Flow and Functional
	Connectivity in Pontine Infarction: A Longitudinal {MRI} Study",
	author   = "Wei, Ying and Wu, Luobing and Wang, Yingying and Liu, Jingchun
	and Miao, Peifang and Wang, Kaiyu and Wang, Caihong and Cheng,
	Jingliang",
	abstract = "Abnormal cerebral blood flow (CBF) and resting-state functional
	connectivity (rs-FC) are sensitive biomarkers of disease
	progression and prognosis. This study investigated neural
	underpinnings of motor and cognitive recovery by longitudinally
	studying the changes of CBF and FC in pontine infarction (PI).
	Twenty patients underwent three-dimensional pseudo-continuous
	arterial spin labeling (3D-pcASL), resting-state functional
	magnetic resonance imaging (rs-fMRI) scans, and behavioral
	assessments at 1 week, 1, 3, and 6 months after stroke. Twenty
	normal control (NC) subjects underwent the same examination once.
	First, we investigated CBF changes in the acute stage, and
	longitudinal changes from 1 week to 6 months after PI. Brain
	regions with longitudinal CBF changes were then used as seeds to
	investigate longitudinal FC alterations during the follow-up
	period. Compared with NC, patients in the left PI (LPI) and right
	PI (RPI) groups showed significant CBF alterations in the
	bilateral cerebellum and some supratentorial brain regions at the
	baseline stage. Longitudinal analysis revealed that altered CBF
	values in the right supramarginal (SMG\_R) for the LPI group,
	while the RPI group showed significantly dynamic changes of CBF
	in the left calcarine sulcus (CAL\_L), middle occipital gyrus
	(MOG\_L), and right supplementary motor area (SMA\_R). Using the
	SMG\_R as the seed in the LPI group, FC changes were found in the
	MOG\_L, middle temporal gyrus (MTG\_L), and prefrontal lobe
	(IFG\_L). Correlation analysis showed that longitudinal CBF
	changes in the SMG\_R and FC values between the SMG\_R and MOG\_L
	were associated with motor and memory scores in the LPI group,
	and longitudinal CBF changes in the CAL\_L and SMA\_R were
	related to memory and motor recovery in the RPI group. These
	longitudinal CBF and accompany FC alterations may provide
	insights into the neural mechanism underlying functional recovery
	after PI, including that of motor and cognitive functions.",
	journal  = "Front. Aging Neurosci.",
	volume   =  12,
	pages    = "577899",
	month    =  nov,
	year     =  2020,
	keywords = "cerebral blood flow; cognition; functional connectivity; motor;
	pontine infarction",
	language = "en"
}

@ARTICLE{Adam2020-jk,
	title    = "Functional reorganization during the recovery of contralesional
	target selection deficits after prefrontal cortex lesions in
	macaque monkeys",
	author   = "Adam, Ramina and Johnston, Kevin and Menon, Ravi S and Everling,
	Stefan",
	abstract = "Visual extinction has been characterized by the failure to
	respond to a visual stimulus in the contralesional hemifield when
	presented simultaneously with an ipsilesional stimulus (Corbetta
	and Shulman, 2011). Unilateral damage to the macaque
	frontoparietal cortex commonly leads to deficits in
	contralesional target selection that resemble visual extinction.
	Recently, we showed that macaque monkeys with unilateral lesions
	in the caudal prefrontal cortex (PFC) exhibited contralesional
	target selection deficits that recovered over 2-4 months (Adam et
	al., 2019). Here, we investigated the longitudinal changes in
	functional connectivity (FC) of the frontoparietal network after
	a small or large right caudal PFC lesion in four macaque monkeys.
	We collected ultra-high field resting-state fMRI at 7-T before
	the lesion and at weeks 1-16 post-lesion and compared the
	functional data with behavioural performance on a free-choice
	saccade task. We found that the pattern of frontoparietal network
	FC changes depended on lesion size, such that the recovery of
	contralesional extinction was associated with an initial increase
	in network FC that returned to baseline in the two small lesion
	monkeys, whereas FC continued to increase throughout recovery in
	the two monkeys with a larger lesion. We also found that the FC
	between contralesional dorsolateral PFC and ipsilesional parietal
	cortex correlated with behavioural recovery and that the
	contralesional dorsolateral PFC showed increasing degree
	centrality with the frontoparietal network. These findings
	suggest that both the contralesional and ipsilesional hemispheres
	play an important role in the recovery of function. Importantly,
	optimal compensation after large PFC lesions may require greater
	recruitment of distant and intact areas of the frontoparietal
	network, whereas recovery from smaller lesions was supported by a
	normalization of the functional network.",
	journal  = "Neuroimage",
	volume   =  207,
	pages    = "116339",
	month    =  feb,
	year     =  2020,
	keywords = "Endothelin-1; Functional connectivity; Monkeys; Resting-state
	fMRI; Visual extinction",
	language = "en"
}

@ARTICLE{Murata2015-ss,
	title    = "Temporal plasticity involved in recovery from manual dexterity
	deficit after motor cortex lesion in macaque monkeys",
	author   = "Murata, Yumi and Higo, Noriyuki and Hayashi, Takuya and
	Nishimura, Yukio and Sugiyama, Yoko and Oishi, Takao and Tsukada,
	Hideo and Isa, Tadashi and Onoe, Hirotaka",
	abstract = "The question of how intensive motor training restores motor
	function after brain damage or stroke remains unresolved. Here we
	show that the ipsilesional ventral premotor cortex (PMv) and
	perilesional primary motor cortex (M1) of rhesus macaque monkeys
	are involved in the recovery of manual dexterity after a lesion
	of M1. A focal lesion of the hand digit area in M1 was made by
	means of ibotenic acid injection. This lesion initially caused
	flaccid paralysis in the contralateral hand but was followed by
	functional recovery of hand movements, including precision grip,
	during the course of daily postlesion motor training. Brain
	imaging of regional cerebral blood flow by means of H2
	(15)O-positron emission tomography revealed enhanced activity of
	the PMv during the early postrecovery period and increased
	functional connectivity within M1 during the late postrecovery
	period. The causal role of these areas in motor recovery was
	confirmed by means of pharmacological inactivation by muscimol
	during the different recovery periods. These findings indicate
	that, in both the remaining primary motor and premotor cortical
	areas, time-dependent plastic changes in neural activity and
	connectivity are involved in functional recovery from the motor
	deficit caused by the M1 lesion. Therefore, it is likely that the
	PMv, an area distant from the core of the lesion, plays an
	important role during the early postrecovery period, whereas the
	perilesional M1 contributes to functional recovery especially
	during the late postrecovery period.",
	journal  = "J. Neurosci.",
	volume   =  35,
	number   =  1,
	pages    = "84--95",
	month    =  jan,
	year     =  2015,
	keywords = "brain activation; functional compensation; macaque monkey;
	precision grip; primate",
	language = "en"
}

@ARTICLE{Hillis2002-dz,
	title    = "Subcortical aphasia and neglect in acute stroke: the role of
	cortical hypoperfusion",
	author   = "Hillis, A E and Wityk, R J and Barker, P B and Beauchamp, N J and
	Gailloud, P and Murphy, K and Cooper, O and Metter, E J",
	abstract = "We have hypothesized that most cases of aphasia or hemispatial
	neglect due to acute, subcortical infarct can be accounted for by
	concurrent cortical hypoperfusion. To test this hypothesis, we
	demonstrate: (i) that pure subcortical infarctions are associated
	with cortical hypoperfusion in subjects with aphasia/neglect;
	(ii) that reversal of cortical hypoperfusion is associated with
	resolution of the aphasia; and (iii) that aphasia/neglect
	strongly predicts cortical ischaemia and/or hypoperfusion. We
	prospectively evaluated a consecutive series of 115 patients who
	presented within 24 h of onset or progression of stroke symptoms,
	with MRI sequences including diffusion weighted imaging (DWI) and
	perfusion weighted imaging (PWI), and detailed testing for
	aphasia or hemispatial neglect. The association between aphasia
	or neglect and cortical infarct (or dense ischaemia) on DWI and
	cortical hypoperfusion indicated by PWI, was evaluated with
	chi-squared analyses. Fisher exact tests were used for analyses
	with small samples. Cases of DWI lesion restricted to subcortical
	white matter and/or grey matter structures (n = 44) were examined
	for the presence of aphasia or neglect, and for the presence of
	cortical hypoperfusion. In addition, subjects who received
	intervention to restore perfusion were evaluated with DWI, PWI,
	and cognitive tests before and after intervention. Finally, the
	positive predictive value of the cognitive deficits for
	identifying cortical abnormalities on DWI and PWI were calculated
	from all patients. Of the subjects with only subcortical lesions
	on DWI in this study (n = 44), all those who had aphasia or
	neglect showed concurrent cortical hypoperfusion. Among the
	patients who received intervention that successfully restored
	cortical perfusion, 100\% (six out of six) showed immediate
	resolution of aphasia. In the 115 patients, aphasia and neglect
	were much more strongly associated with cortical hypoperfusion
	(chi(2) = 57.3 for aphasia; chi(2) = 28.7 for neglect; d.f. = 1;
	P < 0.000001 for each), than with cortical infarct/ischaemia on
	DWI (chi(2) = 8.5 for aphasia; chi(2) = 9.7 for neglect; d.f. =
	1; P < 0.005 for each). Aphasia showed a much higher positive
	predictive value for cortical abnormality on PWI (95\%) than on
	DWI (62\%), as did neglect (100\% positive predictive value for
	PWI versus 74\% for DWI). From these data we conclude that
	aphasia and neglect due to acute subcortical stroke can be
	largely explained by cortical hypoperfusion.",
	journal  = "Brain",
	volume   =  125,
	number   = "Pt 5",
	pages    = "1094--1104",
	month    =  may,
	year     =  2002,
	language = "en"
}

@ARTICLE{Finn2015-er,
	title    = "Functional connectome fingerprinting: identifying individuals
	using patterns of brain connectivity",
	author   = "Finn, Emily S and Shen, Xilin and Scheinost, Dustin and
	Rosenberg, Monica D and Huang, Jessica and Chun, Marvin M and
	Papademetris, Xenophon and Constable, R Todd",
	abstract = "Functional magnetic resonance imaging (fMRI) studies typically
	collapse data from many subjects, but brain functional
	organization varies between individuals. Here we establish that
	this individual variability is both robust and reliable, using
	data from the Human Connectome Project to demonstrate that
	functional connectivity profiles act as a 'fingerprint' that can
	accurately identify subjects from a large group. Identification
	was successful across scan sessions and even between task and
	rest conditions, indicating that an individual's connectivity
	profile is intrinsic, and can be used to distinguish that
	individual regardless of how the brain is engaged during imaging.
	Characteristic connectivity patterns were distributed throughout
	the brain, but the frontoparietal network emerged as most
	distinctive. Furthermore, we show that connectivity profiles
	predict levels of fluid intelligence: the same networks that were
	most discriminating of individuals were also most predictive of
	cognitive behavior. Results indicate the potential to draw
	inferences about single subjects on the basis of functional
	connectivity fMRI.",
	journal  = "Nat. Neurosci.",
	volume   =  18,
	number   =  11,
	pages    = "1664--1671",
	month    =  nov,
	year     =  2015,
	language = "en"
}

@ARTICLE{Winship2009-af,
	title    = "Remapping the somatosensory cortex after stroke: insight from
	imaging the synapse to network",
	author   = "Winship, Ian R and Murphy, Timothy H",
	abstract = "Together, thousands of neurons with similar function make up
	topographically oriented sensory cortex maps that represent
	contralateral body parts. Although this is an accepted model for
	the adult cortex, whether these same rules hold after
	stroke-induced damage is unclear. After stroke, sensory
	representations damaged by stroke remap onto nearby surviving
	neurons. Here, we review the process of sensory remapping after
	stroke at multiple levels ranging from the initial damage to
	synapses, to their rewiring and function in intact sensory
	circuits. We introduce a new approach using in vivo 2-photon
	calcium imaging to determine how the response properties of
	individual somatosensory cortex neurons are altered during
	remapping. One month after forelimb-area stroke, normally highly
	limb-selective neurons in surviving peri-infarct areas exhibit
	remarkable flexibility and begin to process sensory stimuli from
	multiple limbs as remapping proceeds. Two months after stroke,
	neurons within remapped regions develop a stronger response
	preference. Thus, remapping is initiated by surviving neurons
	adopting new roles in addition to their usual function. Later in
	recovery, these remapped forelimb-responsive neurons become more
	selective, but their new topographical representation may
	encroach on map territories of neurons that process sensory
	stimuli from other body parts. Neurons responding to multiple
	limbs may reflect a transitory phase in the progression from
	their involvement in one sensorimotor function to a new function
	that replaces processing lost due to stroke.",
	journal  = "Neuroscientist",
	volume   =  15,
	number   =  5,
	pages    = "507--524",
	month    =  oct,
	year     =  2009,
	language = "en"
}

@ARTICLE{Corbetta2005-ra,
	title    = "Neural basis and recovery of spatial attention deficits in
	spatial neglect",
	author   = "Corbetta, Maurizio and Kincade, Michelle J and Lewis, Chris and
	Snyder, Abraham Z and Sapir, Ayelet",
	abstract = "The syndrome of spatial neglect is typically associated with
	focal injury to the temporoparietal or ventral frontal cortex.
	This syndrome shows spontaneous partial recovery, but the neural
	basis of both spatial neglect and its recovery is largely
	unknown. We show that spatial attention deficits in neglect
	(rightward bias and reorienting) after right frontal damage
	correlate with abnormal activation of structurally intact dorsal
	and ventral parietal regions that mediate related attentional
	operations in the normal brain. Furthermore, recovery of these
	attention deficits correlates with the restoration and
	rebalancing of activity within these regions. These results
	support a model of recovery based on the re-weighting of activity
	within a distributed neuronal architecture, and they show that
	behavioral deficits depend not only on structural changes at the
	locus of injury, but also on physiological changes in distant but
	functionally related brain areas.",
	journal  = "Nat. Neurosci.",
	volume   =  8,
	number   =  11,
	pages    = "1603--1610",
	month    =  nov,
	year     =  2005,
	language = "en"
}

@ARTICLE{Kuceyeski2013-nk,
	title    = "The Network Modification ({NeMo}) Tool: elucidating the effect of
	white matter integrity changes on cortical and subcortical
	structural connectivity",
	author   = "Kuceyeski, Amy and Maruta, Jun and Relkin, Norman and Raj, Ashish",
	abstract = "Accurate prediction of brain dysfunction caused by disease or
	injury requires the quantification of resultant neural
	connectivity changes compared with the normal state. There are
	many methods with which to assess anatomical changes in
	structural or diffusion magnetic resonance imaging, but most
	overlook the topology of white matter (WM) connections that make
	up the healthy brain network. Here, a new neuroimaging software
	pipeline called the Network Modification (NeMo) Tool is presented
	that associates alterations in WM integrity with expected changes
	in neural connectivity between gray matter regions. The NeMo Tool
	uses a large reference set of healthy tractograms to assess
	implied network changes arising from a particular pattern of WM
	alteration on a region- and network-wise level. In this way, WM
	integrity changes can be extrapolated to the cortices and deep
	brain nuclei, enabling assessment of functional and cognitive
	alterations. Unlike current techniques that assess network
	dysfunction, the NeMo tool does not require tractography in
	pathological brains for which the algorithms may be unreliable or
	diffusion data are unavailable. The versatility of the NeMo Tool
	is demonstrated by applying it to data from patients with
	Alzheimer's disease, fronto-temporal dementia, normal pressure
	hydrocephalus, and mild traumatic brain injury. This tool fills a
	gap in the quantitative neuroimaging field by enabling an
	investigation of morphological and functional implications of
	changes in structural WM integrity.",
	journal  = "Brain Connect.",
	volume   =  3,
	number   =  5,
	pages    = "451--463",
	year     =  2013,
	language = "en"
}

@ARTICLE{Liegeois2020-ua,
	title     = "Revisiting correlation-based functional connectivity and its
	relationship with structural connectivity",
	author    = "Li{\'e}geois, Rapha{\"e}l and Santos, Augusto and Matta,
	Vincenzo and Van De Ville, Dimitri and Sayed, Ali H",
	abstract  = "Patterns of brain structural connectivity (SC) and functional
	connectivity (FC) are known to be related. In SC-FC comparisons,
	FC has classically been evaluated from correlations between
	functional time series, and more recently from partial
	correlations or their unnormalized version encoded in the
	precision matrix. The latter FC metrics yields more meaningful
	comparisons to SC because they capture ?direct? statistical
	dependencies, i.e., discarding the effects of mediators, but
	their use has been limited because of estimation issues. With
	the rise of high-quality and large neuroimaging datasets, we
	revisit relevance of different FC metrics in the context of
	SC-FC comparisons. Using data from 100 unrelated HCP subjects,
	we first explore the amount of functional data required to
	reliably estimate various FC metrics. We find that
	precision-based FC yields a better match to SC than
	correlation-based FC when using 5 minutes of functional data or
	more. Finally, using a linear model linking SC and FC, we show
	that the SC-FC match can be used to further interrogate various
	aspects of brain structure and function such as the timescales
	of functional dynamics in different resting-state networks or
	the intensity of anatomical self-connections.",
	journal   = "Network Neuroscience",
	publisher = "MIT Press",
	pages     = "1--17",
	month     =  aug,
	year      =  2020
}

@ARTICLE{Brown2009-jn,
	title    = "In vivo voltage-sensitive dye imaging in adult mice reveals that
	somatosensory maps lost to stroke are replaced over weeks by new
	structural and functional circuits with prolonged modes of
	activation within both the peri-infarct zone and distant sites",
	author   = "Brown, Craig E and Aminoltejari, Khatereh and Erb, Heidi and
	Winship, Ian R and Murphy, Timothy H",
	abstract = "After brain damage such as stroke, topographically organized
	sensory and motor cortical representations remap onto adjacent
	surviving tissues. It is conceivable that cortical remapping is
	accomplished by changes in the temporal precision of sensory
	processing and regional connectivity in the cortex. To understand
	how the adult cortex remaps and processes sensory signals during
	stroke recovery, we performed in vivo imaging of sensory-evoked
	changes in membrane potential, as well as multiphoton imaging of
	dendrite structure and tract tracing. In control mice, forelimb
	stimulation evoked a brief depolarization in forelimb cortex that
	quickly propagated to, and dissipated within, adjacent
	motor/hindlimb areas (<100 ms). One week after forelimb cortex
	stroke, the cortex was virtually unresponsive to tactile forelimb
	stimulation. After 8 weeks recovery, forelimb-evoked
	depolarizations reemerged with a characteristic pattern in which
	responses began within surviving portions of forelimb cortex (<20
	ms after stimulation) and then spread horizontally into
	neighboring peri-infarct motor/hindlimb areas in which
	depolarization persisted 300-400\% longer than controls. These
	uncharacteristically prolonged responses were not limited to the
	remapped peri-infarct zone and included distant posteromedial
	retrosplenial cortex, millimeters from the stroke. Structurally,
	the remapped peri-infarct area selectively exhibited high levels
	of dendritic spine turnover, shared more connections with
	retrosplenial cortex and striatum, and lost inputs from lateral
	somatosensory cortical regions. Our findings demonstrate that
	sensory remapping during stroke recovery is accompanied by the
	development of prolonged sensory responses and new structural
	circuits in both the peri-infarct zone as well as more distant
	sites.",
	journal  = "J. Neurosci.",
	volume   =  29,
	number   =  6,
	pages    = "1719--1734",
	month    =  feb,
	year     =  2009,
	language = "en"
}

@ARTICLE{Wodeyar2020-kz,
	title     = "Damage to the structural connectome reflected in resting-state
	{fMRI} functional connectivity",
	author    = "Wodeyar, Anirudh and Cassidy, Jessica M and Cramer, Steven C and
	Srinivasan, Ramesh",
	abstract  = "The relationship between structural and functional connectivity
	has been mostly examined in intact brains. Fewer studies have
	examined how differences in structure as a result of injury
	alters function. In this study we analyzed the relationship of
	structure to function across patients with stroke among whom
	infarcts caused heterogenous structural damage. We estimated
	relationships between distinct brain regions of interest (ROIs)
	from functional MRI in two pipelines. In one analysis pipeline,
	we measured functional connectivity by using correlation and
	partial correlation between 114 cortical ROIs. We found
	fMRI-BOLD partial correlation was altered at more edges as a
	function of the structural connectome (SC) damage, relative to
	the correlation. In a second analysis pipeline, we limited our
	analysis to fMRI correlations between pairs of voxels for which
	we possess SC information. We found that voxel-level functional
	connectivity showed the effect of structural damage that we
	could not see when examining correlations between ROIs. Further,
	the effects of structural damage on functional connectivity are
	consistent with a model of functional connectivity, diffusion,
	which expects functional connectivity to result from activity
	spreading over multiple edge anatomical paths.",
	journal   = "Network Neuroscience",
	publisher = "MIT Press",
	pages     = "1--22",
	month     =  jul,
	year      =  2020
}

@ARTICLE{Osmanlioglu2019-ao,
	title    = "System-level matching of structural and functional connectomes in
	the human brain",
	author   = "Osmanl{\i}o{\u g}lu, Yusuf and Tun{\c c}, Birkan and Parker, Drew
	and Elliott, Mark A and Baum, Graham L and Ciric, Rastko and
	Satterthwaite, Theodore D and Gur, Raquel E and Gur, Ruben C and
	Verma, Ragini",
	abstract = "The brain can be considered as an information processing network,
	where complex behavior manifests as a result of communication
	between large-scale functional systems such as visual and default
	mode networks. As the communication between brain regions occurs
	through underlying anatomical pathways, it is important to define
	a ``traffic pattern'' that properly describes how the regions
	exchange information. Empirically, the choice of the traffic
	pattern can be made based on how well the functional connectivity
	between regions matches the structural pathways equipped with
	that traffic pattern. In this paper, we present a multimodal
	connectomics paradigm utilizing graph matching to measure
	similarity between structural and functional connectomes (derived
	from dMRI and fMRI data) at node, system, and connectome level.
	Through an investigation of the brain's structure-function
	relationship over a large cohort of 641 healthy developmental
	participants aged 8-22 years, we demonstrate that communicability
	as the traffic pattern describes the functional connectivity of
	the brain best, with large-scale systems having significant
	agreement between their structural and functional connectivity
	patterns. Notably, matching between structural and functional
	connectivity for the functionally specialized modular systems
	such as visual and motor networks are higher as compared to other
	more integrated systems. Additionally, we show that the negative
	functional connectivity between the default mode network (DMN)
	and motor, frontoparietal, attention, and visual networks is
	significantly associated with its underlying structural
	connectivity, highlighting the counterbalance between functional
	activation patterns of DMN and other systems. Finally, we
	investigated sex difference and developmental changes in brain
	and observed that similarity between structure and function
	changes with development.",
	journal  = "Neuroimage",
	volume   =  199,
	pages    = "93--104",
	month    =  oct,
	year     =  2019,
	keywords = "Connectomics; Large-scale systems; MRI; Network analysis;
	Structure-function matching",
	language = "en"
}

@ARTICLE{Corbetta2015-ez,
	title    = "Common behavioral clusters and subcortical anatomy in stroke",
	author   = "Corbetta, Maurizio and Ramsey, Lenny and Callejas, Alicia and
	Baldassarre, Antonello and Hacker, Carl D and Siegel, Joshua S
	and Astafiev, Serguei V and Rengachary, Jennifer and Zinn,
	Kristina and Lang, Catherine E and Connor, Lisa Tabor and
	Fucetola, Robert and Strube, Michael and Carter, Alex R and
	Shulman, Gordon L",
	abstract = "A long-held view is that stroke causes many distinct neurological
	syndromes due to damage of specialized cortical and subcortical
	centers. However, it is unknown if a syndrome-based description
	is helpful in characterizing behavioral deficits across a large
	number of patients. We studied a large prospective sample of
	first-time stroke patients with heterogeneous lesions at 1-2
	weeks post-stroke. We measured behavior over multiple domains and
	lesion anatomy with structural MRI and a probabilistic atlas of
	white matter pathways. Multivariate methods estimated the
	percentage of behavioral variance explained by structural damage.
	A few clusters of behavioral deficits spanning multiple functions
	explained neurological impairment. Stroke topography was
	predominantly subcortical, and disconnection of white matter
	tracts critically contributed to behavioral deficits and their
	correlation. The locus of damage explained more variance for
	motor and language than memory or attention deficits. Our
	findings highlight the need for better models of white matter
	damage on cognition.",
	journal  = "Neuron",
	volume   =  85,
	number   =  5,
	pages    = "927--941",
	month    =  mar,
	year     =  2015,
	language = "en"
}

@ARTICLE{Wang2010-or,
	title    = "Dynamic functional reorganization of the motor execution network
	after stroke",
	author   = "Wang, Liang and Yu, Chunshui and Chen, Hai and Qin, Wen and He,
	Yong and Fan, Fengmei and Zhang, Yujin and Wang, Moli and Li,
	Kuncheng and Zang, Yufeng and Woodward, Todd S and Zhu, Chaozhe",
	abstract = "Numerous studies argue that cortical reorganization may
	contribute to the restoration of motor function following stroke.
	However, the evolution of changes during the post-stroke
	reorganization has been little studied. This study sought to
	identify dynamic changes in the functional organization,
	particularly topological characteristics, of the motor execution
	network during the stroke recovery process. Ten patients (nine
	male and one female) with subcortical infarctions were assessed
	by neurological examination and scanned with resting-state
	functional magnetic resonance imaging across five consecutive
	time points in a single year. The motor execution network of each
	subject was constructed using a functional connectivity matrix
	between 21 brain regions and subsequently analysed using graph
	theoretical approaches. Dynamic changes in topological
	configuration of the network during the process of recovery were
	evaluated by a mixed model. We found that the motor execution
	network gradually shifted towards a random mode during the
	recovery process, which suggests that a less optimized
	reorganization is involved in regaining function in the affected
	limbs. Significantly increased regional centralities within the
	network were observed in the ipsilesional primary motor area and
	contralesional cerebellum, whereas the ipsilesional cerebellum
	showed decreased regional centrality. Functional connectivity to
	these brain regions demonstrated consistent alterations over
	time. Notably, these measures correlated with different clinical
	variables, which provided support that the findings may reflect
	the adaptive reorganization of the motor execution network in
	stroke patients. In conclusion, the study expands our
	understanding of the spectrum of changes occurring in the brain
	after stroke and provides a new avenue for investigating
	lesion-induced network plasticity.",
	journal  = "Brain",
	volume   =  133,
	number   = "Pt 4",
	pages    = "1224--1238",
	month    =  apr,
	year     =  2010,
	language = "en"
}

@ARTICLE{Lu2011-ow,
	title    = "Focal pontine lesions provide evidence that intrinsic functional
	connectivity reflects polysynaptic anatomical pathways",
	author   = "Lu, Jie and Liu, Hesheng and Zhang, Miao and Wang, Danhong and
	Cao, Yanxiang and Ma, Qingfeng and Rong, Dongdong and Wang,
	Xiaoyi and Buckner, Randy L and Li, Kuncheng",
	abstract = "Intrinsic functional connectivity detected by functional MRI
	(fMRI) provides a useful but indirect approach to study the
	organization of human brain systems. An unresolved question is
	whether functional connectivity measured by resting-state fMRI
	reflects anatomical connections. In this study, we used the
	well-characterized anatomy of cerebrocerebellar circuits to
	directly test whether intrinsic functional connectivity is
	associated with an anatomic pathway. Eleven first-episode stroke
	patients were scanned five times during a period of 6 months, and
	11 healthy control subjects were scanned three times within 1
	month. In patients with right pontine strokes, the functional
	connectivity between the right motor cortex and the left
	cerebellum was selectively reduced. This connectivity pattern was
	reversed in patients with left pontine strokes. Although factors
	beyond anatomical connectivity contribute to fMRI measures of
	functional correlation, these results provide direct evidence
	that functional connectivity depends on intact connections within
	a specific polysynaptic pathway.",
	journal  = "J. Neurosci.",
	volume   =  31,
	number   =  42,
	pages    = "15065--15071",
	month    =  oct,
	year     =  2011,
	language = "en"
}
